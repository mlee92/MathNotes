% !TEX TS-program = pdflatex
% !TEX encoding = UTF-8 Unicode

% This is a simple template for a LaTeX document using the "article" class.
% See "book", "report", "letter" for other types of document.

\documentclass[11pt]{article} % use larger type; default would be 10pt

\usepackage[utf8]{inputenc} % set input encoding (not needed with XeLaTeX)


%%% Examples of Article customizations
% These packages are optional, depending whether you want the features they provide.
% See the LaTeX Companion or other references for full information.

%%% PAGE DIMENSIONS
\usepackage{geometry} % to change the page dimensions
\geometry{a4paper} % or letterpaper (US) or a5paper or....
% \geometry{margin=2in} % for example, change the margins to 2 inches all round
% \geometry{landscape} % set up the page for landscape
%   read geometry.pdf for detailed page layout information

\usepackage{graphicx} % support the \includegraphics command and options

% \usepackage[parfill]{parskip} % Activate to begin paragraphs with an empty line rather than an indent

%%% PACKAGES
\usepackage{booktabs} % for much better looking tables
\usepackage{array} % for better arrays (eg matrices) in maths
\usepackage{paralist} % very flexible & customisable lists (eg. enumerate/itemize, etc.)
\usepackage{verbatim} % adds environment for commenting out blocks of text & for better verbatim
\usepackage{subfig} % make it possible to include more than one captioned figure/table in a single float
% These packages are all incorporated in the memoir class to one degree or another...

%%% HEADERS & FOOTERS
\usepackage{fancyhdr} % This should be set AFTER setting up the page geometry
\pagestyle{fancy} % options: empty , plain , fancy
\renewcommand{\headrulewidth}{0pt} % customise the layout...
\lhead{}\chead{}\rhead{}
\lfoot{}\cfoot{\thepage}\rfoot{}

\usepackage{amssymb}

%%% SECTION TITLE APPEARANCE
\usepackage{sectsty}
\allsectionsfont{\sffamily\mdseries\upshape} % (See the fntguide.pdf for font help)
% (This matches ConTeXt defaults)

%%% ToC (table of contents) APPEARANCE
\usepackage[nottoc,notlof,notlot]{tocbibind} % Put the bibliography in the ToC
\usepackage[titles,subfigure]{tocloft} % Alter the style of the Table of Contents
\renewcommand{\cftsecfont}{\rmfamily\mdseries\upshape}
\renewcommand{\cftsecpagefont}{\rmfamily\mdseries\upshape} % No bold!

%%% END Article customizations

%%% Theorem Environment
\newtheorem{theorem}{Theorem}[section]
\newtheorem{lemma}[theorem]{Lemma}
\newtheorem{proposition}[theorem]{Proposition}
\newtheorem{corollary}[theorem]{Corollary}

\newenvironment{proof}[1][Proof]{\begin{trivlist}
\item[\hskip \labelsep {\bfseries #1}]}{\end{trivlist}}
\newenvironment{definition}[1][Definition]{\begin{trivlist}
\item[\hskip \labelsep {\bfseries #1}]}{\end{trivlist}}
\newenvironment{example}[1][Example]{\begin{trivlist}
\item[\hskip \labelsep {\bfseries #1}]}{\end{trivlist}}
\newenvironment{axiom}[1][Axiom]{\begin{trivlist}
\item[\hskip \labelsep {\bfseries #1}]}{\end{trivlist}}
\newenvironment{formula}[1][Formula]{\begin{trivlist}
\item[\hskip \labelsep {\bfseries #1}]}{\end{trivlist}}
\newenvironment{remark}[1][Remark]{\begin{trivlist}
\item[\hskip \labelsep {\bfseries #1}]}{\end{trivlist}}
\newenvironment{property}[1][Property]{\begin{trivlist}
\item[\hskip \labelsep {\bfseries #1}]}{\end{trivlist}}
\newenvironment{assignment}[1][Assignment]{\begin{trivlist}
\item[\hskip \labelsep {\bfseries #1}]}{\end{trivlist}}

\newcommand{\qed}{\nobr_eak \ifvmode \relax \else
      \ifdim\lastskip<1.5em \hskip-\lastskip
      \hskip1.5em plus0em minus0.5em \fi \nobreak
      \vrule height0.75em width0.5em depth0.25em\fi}
%%% END Theorem Environment

%%% The "real" document content comes below...

\title{Notes from Sheldon Ross' A First Course in Probability}
\author{Michael Lee}


\begin{document}
\maketitle


\section{Chapter 1: Combinatorial Analysis}

\begin{formula}
	There are ${ {n - 1} \choose {r - 1}}$ distinct positive integer-valued vectors ($X_1, X_2, \ldots, X_r$) satsify $X_1 + X_2 + \ldots + X_r = n$.   
\end{formula}

\begin{formula}
		There are ${ {n + r - 1} \choose {n}}$ distinct non-negative integer-valued vectors ($X_1, X_2, \ldots, X_r$) satsify $X_1 + X_2 + \ldots + X_r = n$.   
\end{formula}

\section{Axioms of Probability}

\subsection{Sample Space and Events}

\begin{definition}
	This set of all possible outcomes of an experiement is known as the {\it sample space} of the experiment and is denoted by {\bf S}. If the experiment consists of $n$ independent events of $m$ possibilities, then the sample space consists of $n \ldots m$ points. {\bf S} is denoted by a set characterised by the vector $n$ variables, followed by the possible values of $n$. Any subset $E$ of the sample space is known as an {\it event}.
\end{definition}

\begin{definition}
	The event ${E \cup F}$ is called the {\it union} of the even $E$ and the event $F$. The event ${E \cap F}$ or ${EF}$ is called the {\it intersection} of events $E$ and $F$ to consist of all outcomes that are both in $E$ and $F$. 
\end{definition}

\begin{definition}
	The null event $\emptyset$ refers to the event consisting of no points. If $EF = \emptyset$, then $E$ and $F$ sare said to be {\it mutually exclusive}. 
\end{definition}

\begin{definition}
	If events {$E_1, E_2, \ldots, E_n$} are events, the union of these events, denoted by {$\cup_{n = 1}^{\infty}$} is defined to be that event which consists of all points that in {$E_n$} for at least one value of {$n = 1, 2, \ldots$}. Similarly, the intersection of the events $E_n$ denoted by $\cap_{n =1}^{\infty}$ is defined to be the event consisting fo those points that are in all of the events $E_n, n = 1, 2, \ldots$. 
\end{definition}

\begin{definition}
	For any event $E$, we define the new event $E^c$ as the {\it complement} of $E$ to cosnist of all points in the sample space $S$ that are not in $E$. $E^c$ occurs if and only if $E$ does not occur. 
\end{definition}

\begin{definition}
	For any two events $E$ and $F$, if all of the points $E$ and $F$, if all of the points in $E$ are also in $F$, then we say that $E$ is cotnained in $F$ and write $E \subset F$ or$ F \supset E$. Therefore, if $E \subset F$, the occurrence of $E$ will necessarily imply the occurrence of $F$. IF $E \subset F$ and $F \subset E$, $E = F$.
\end{definition}

\begin{formula} 
	\mbox{} \\
	{\bf Set Theory Laws} \\
	Commutative Law: $E \cup F = F \cup$ AND $EF = FE$ \\
	Associative Law $(E \cup F) \cup G = E \cup (F \cup G) $ AND $ (EF)G = E(FG)$ \\
	Distributive Law $(E \cup F) G = EG \cup FG$ AND $EF \cup G = (E \cup G) = (E \cup G)(F \cup G)$


	\mbox{} \\
	{\bf DeMorgan's Laws} \\
	$(\cup_{i = 1}^{n}{E_i})^c = \cap_{i = 1}^{n}{E_i}^c$ \\
	$(\cap_{i =1}^{n}{E_i})^c = \cup_{i = 1}^{n}{E_i}^c$

\end{formula}

\subsection{Axioms of Probability}

\begin{definition}
	One way of defining the probabilityo f an event is in terms of its relative frequency. That is $P(E)$ is defined as the limiting percentage of time that $E$ occurs. 
\end{definition}

\begin{axiom}
	$ 0 \leq P(E) \leq 1$ \\
The probability that the outcome of the experiment is a point in $E$ is some number between 0 and 1.
\end{axiom}

\begin{axiom}
	$ P(S) = 1 $ \\
The outcome will be a point in the sample space $S$. 
\end{axiom}

\begin{axiom}
	For any sequence of mutually exclusive events $E_1, E_2, \ldots $ (that is events for which $E_iE_j = \emptyset$ when $i \neq j)$
	{\center
		$P(\cup_{i = 1}^{\infty}) = \Sigma_{i =1}^{\infty}{P(E_i)}$ \\
	} P(E) is the probability of the event $E$. For ar any sequence of mutually exclusive events the probability of at least one of these events occurring is just the sum of their respective probabilities. 
\end{axiom}

\end{document}

